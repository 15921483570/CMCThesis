% !TeX document-id = {a9da8c09-8ca2-42d8-97aa-816e53d9f143}
% !TeX TXS-program:compile = txs:///latexmk/{}[-xelatex -synctex=1 -interaction=nonstopmode -silent %.tex]
\documentclass[hideanswer=false,
enfont=newtxtext,
zhfont=empty,
mathfont=newtxmath,
]{CMCThesis}

% 是否隐藏答案, hideanswer = false,true
% 英文字体选择, enfont     = newtxtext,noto,empty
% 中文字体设置,  zhfont     = zhnoto,origin,empty
% 数学字体设置, mathfont   = newtxmath,unicode-math,mtpro2,empty
% 选择了 empty 方案的可以使用通用的方法自行设置
% 以上选项可以全部留空,调用默认的配置
% noto 无论英文还是中文字体比较全,如果启用它,记得要把字体放在 fonts/ 下面
% origin 是八一的配置方案,开启它就是之前的设置,字体可以安装使用,或者放在当前文件夹或者 fonts/ 下面使用

% newtxtext 和 newtxmath 配
% noto(enfont) 和 zhnoto(zhfont) 配
%\setmainfont{Times New Roman}
%\setmathfont[
%Extension    = .otf          ,
%Path         = fonts/        ,
%BoldFont     = XITSMath-Bold ,
%StylisticSet = 8]{XITSMath-Regular}% only when mathfont=unicode-math,具体字体可以自己选择

\CMCThesissetup{% key = value 设置处
	%
}
\let\leq\leqslant\let\geq\geqslant
\def\sgn{\mathop{\rm sgn}}
\everymath{\displaystyle}

\newcommand{\ee}{\mathrm e}
\newcommand{\dd}{\,\mathrm{d}}
\newcommand{\textop}[1]{\relax\ifmmode\mathop{\text{#1}}\else\text{#1}\fi}
%%%%%%定义两个列格式,数学与非数学模式
\newcolumntype{Y}{>{\centering\arraybackslash$}X<{$}}
\newcolumntype{Z}{>{\centering\arraybackslash}X}
\newcolumntype{L}{>{\raggedright\arraybackslash}X}
\newcolumntype{R}{>{\raggedleft\arraybackslash}X}
%定义绝对值
\newcommand\abs[1]{\left| #1 \right|}
\makeatletter
\newcommand{\rmnum}[1]{\romannumeral #1}
\newcommand{\Rmnum}[1]{\expandafter\@slowromancap\romannumeral #1@}
\makeatother

%%%%%%%%%%%%%%%%%%%%%%%%%%%%%%%%%%%%%%%%%%%%%%

\newcommand{\chaoda}{\fontsize{55pt}{\baselineskip}\selectfont}
\newcommand{\chuhao}{\fontsize{42pt}{\baselineskip}\selectfont}     % 字号设置
\newcommand{\xiaochuhao}{\fontsize{36pt}{\baselineskip}\selectfont} % 字号设置
\newcommand{\yihao}{\fontsize{28pt}{\baselineskip}\selectfont}      % 字号设置
\newcommand{\erhao}{\fontsize{21pt}{\baselineskip}\selectfont}      % 字号设置
\newcommand{\xiaoerhao}{\fontsize{18pt}{\baselineskip}\selectfont}  % 字号设置
\newcommand{\sanhao}{\fontsize{15.75pt}{\baselineskip}\selectfont}  % 字号设置
\newcommand{\xiaosanhao}{\fontsize{15pt}{\baselineskip}\selectfont} % 字号设置
\newcommand{\sihao}{\fontsize{14pt}{\baselineskip}\selectfont}      % 字号设置
\newcommand{\xiaosihao}{\fontsize{12pt}{14pt}\selectfont}           % 字号设置
\newcommand{\wuhao}{\fontsize{10.5pt}{12.6pt}\selectfont}           % 字号设置
\newcommand{\xiaowuhao}{\fontsize{9pt}{11pt}{\baselineskip}\selectfont}   % 字号设置
\newcommand{\liuhao}{\fontsize{7.875pt}{\baselineskip}\selectfont}  % 字号设置
\newcommand{\qihao}{\fontsize{5.25pt}{\baselineskip}\selectfont}    % 字号设置

\everymath{\displaystyle}
\usepackage[thmmarks,amsmath]{ntheorem}
\theoremstyle{nonumberplain}
\theoremheaderfont{\bfseries}
\theorembodyfont{\normalfont}
{
	\theoremstyle{nonumberplain}%不带标号
	\theoremheaderfont{\bfseries}%证明题头加粗
	\theorembodyfont{\normalfont}
	%	\theorembodyfont{\songti}%楷书字体
	\theoremsymbol{\mbox{$\Box$}}%结束以后自动画出一个小方块
	\newtheorem{solution}{解.}%名字叫做"solution",会在题头自动写上证明
}
{
	\theoremstyle{nonumberplain}%不带标号
	\theoremheaderfont{\bfseries}%证明题头加粗
	\theorembodyfont{\normalfont}
	%	\theorembodyfont{\songti}%楷书字体
	\theoremsymbol{\mbox{$\blacksquare$}}%结束以后自动画出一个小黑色方块
	%	\theoremsymbol{\mbox{$\Box$}}%结束以后自动画出一个小方块
	\newtheorem{proof}{证明.}%名字叫做"proof",会在题头自动写上证明
}


\begin{document}
	\CMCThesistitle{
		date= 2018年10月27号 \thinspace 9:00 - 11:30  ,
	}

\addvspace{1\bigskipamount}

\ws{\textbf{填空题}}{( \textbf{本题满分24分,每题6分})\\}\\\\
\wq 设 $\alpha\in\left(0,1\right)$,则$ \lim_{n\rightarrow +\infty}\left(\left(n+1\right)^{\alpha}-n^{\alpha}\right)=$\underline{\hspace{3em}}.\\
	\begin{answer}
	\begin{solution}
等价无穷小$\left(1+x\right)^{\alpha}-1\backsim\alpha x$,得
		\[
		\lim_{n\rightarrow\infty}\left(\left(n+1\right)^{\alpha}-n^{\alpha}\right)=\lim_{n\rightarrow\infty}n^{\alpha}\left(\left(1+1/n\right)^{\alpha}-1\right)=\lim_{n\rightarrow\infty}n^{\alpha}\times\frac{\alpha}{n}=0
		\]
	\end{solution}
	\end{answer}
\wq $\textrm{若曲线}y=f\left(x\right)\textrm{是由}\left\{\begin{array}{l}
x=t+\cos t\\
e^y+ty+\sin t=1\\
\end{array}\right.\textrm{确定,则此曲线在}t=0$ 对应点处的\\
切线方程为\underline{\hspace{3em}}.\\
	\begin{answer}
	\begin{solution}
易知$t=0$处上的曲线为点$(1,0)$,即方程组对$t$求导得
\[
\frac{\mathrm{d}x}{\mathrm{d}t}=1-\sin t\textbf{,}\frac{\mathrm{d}y}{\mathrm{d}t}=-\frac{y+\cos t}{e^y+t}
\]
\[
\Rightarrow\frac{\mathrm{d}y}{\mathrm{d}t}=\frac{\mathrm{d}y}{\mathrm{d}t}/\frac{\mathrm{d}x}{\mathrm{d}t}=-\frac{y+\cos t}{\left(e^y+1\right)\left(1-\sin t\right)}\Rightarrow\frac{\mathrm{d}y}{\mathrm{d}x}|_{t=0}=-1
\]
故曲线在$t=0$对应点处的切线方程为$x+y-1=0$.
	\end{solution}
	\end{answer}

\wq $\upint{\frac{\ln\left(x+\sqrt{1+x^2}\right)}{\left(1+x^2\right)^{\frac{3}{2}}}}\mathrm{d}x=$\underline{\hspace{3em}}.\\
	\begin{answer}
	\begin{solution}
简单的凑微分,如下
\begin{align*}
\upint{\frac{\ln\left(x+\sqrt{1+x^2}\right)}{\left(1+x^2\right)^{\frac{3}{2}}}}\mathrm{d}x&=\upint{\ln\left(x+\sqrt{1+x^2}\right)}\mathrm{d}\left(\frac{x}{\sqrt{1+x^2}}\right)\\
&=\frac{x}{\sqrt{1+x^2}}\ln\left(x+\sqrt{1+x^2}\right)-\upint{\frac{x}{1+x^2}\mathrm{d}x}\\
&=\frac{x}{\sqrt{1+x^2}}\ln\left(x+\sqrt{1+x^2}\right)-\frac{1}{2}\ln\left(1+x^2\right)+C
\end{align*}
	\end{solution}
	\end{answer}

\wq $\lim_{x\rightarrow 0}\frac{1-\cos x\sqrt{\cos 2x}\sqrt[3]{\cos 3x}}{x^2}=$\underline{\hspace{3em}}.\\
	\begin{answer}
	\begin{solution} 由$\cos x= 1-\frac{x^2}{2}+o(x^2),\sqrt{1+z}=1+\frac{z}{2}+o(z)$,即
		\[\sqrt{\cos 2x}= 1-x^2+o(x^2),\quad \sqrt[3]{\cos 3x}= 1-\frac{3x^2}{2}+o(x^2),\quad \sqrt[3]{1+z}=1+\frac{z}{3}+o(z)\]
		即
		\[
		\text{原式}=\lim_{x\rightarrow 0}\frac{1-\left(1-\frac{x^2}{2}+o\left(x^2\right)\right)\left(1-x^2+o\left(x^2\right)\right)\left(1-\frac{3x^2}{2}+o\left(x^2\right)\right)}{x^2}=3
		\]
这题方法除泰勒之外,还可简单的等价无穷小,或拆项,洛必达处理
\begin{align*}
\lim_{x\rightarrow 0}\frac{1-\cos x\sqrt{\cos 2x}\sqrt[3]{\cos 3x}}{x^2}&=\lim_{x\rightarrow 0}\left(\frac{1-\cos x}{x^2}+\frac{\cos x\left(1-\sqrt{\cos 2x}\sqrt[3]{\cos 3x}\right)}{x^2}\right)\\
&=\frac{1}{2}+\lim_{x\rightarrow 0}\frac{1-\sqrt{\cos 2x}\sqrt[3]{\cos 3x}}{x^2}\\
&=\frac{1}{2}+\lim_{x\rightarrow 0}\left(\frac{1-\cos\sqrt{2x}}{x^2}+\frac{\cos\sqrt{2x}\left(1-\sqrt[3]{\cos 3x}\right)}{x^2}\right)\\
&=\frac{1}{2}+\lim_{x\rightarrow 0}\left(\frac{1-\sqrt{1+\left(\cos 2x-1\right)}}{x^2}+\frac{1-\sqrt[3]{1+\left(\cos 3x-1\right)}}{x^2}\right)\\
&=\frac{1}{2}+\lim_{x\rightarrow 0}\frac{1-\cos 2x}{2x^2}+\lim_{x\rightarrow 0}\frac{1-\cos 3x}{3x^2}\\
&=3
\end{align*}
	\end{solution}
	\end{answer}
\ws { \textbf{解答题}}{(\textbf{ 本题满分8分})\\}\\\\
设函数$f(t)$在$t\ne 0$时一阶连续可导,且$f(1)=0$,求函数$f(x^2-y^2)$,使得曲线积分$\displaystyle \upint_L{y\left(2-f\left(x^2-y^2\right)\right)}\mathrm {d}x+xf\left(x^2-y^2\right)\mathrm{d}y
$与路径无关,其中$L$为任一不与直线$y=\pm x$相交的分段光滑闭曲线.
	\begin{answer}
	\begin{solution}
记$\left\{\begin{array}{l}
P\left(x,y\right)=y\left(2-f\left(x^2-y^2\right)\right)\\
Q\left(x,y\right)=x+xf\left(x^2-y^2\right)\\
\end{array}\right. $,于是
\[
\left\{\begin{array}{l}
\frac{\partial P\left(x,y\right)}{\partial y}=2-f\left(x^2-y^2\right)+2y^2f'\left(x^2-y^2\right)\\
\frac{\partial Q\left(x,y\right)}{\partial x}=f\left(x^2+y^2\right)+2x^2f'\left(x^2-y^2\right)\\
\end{array}\right. 
\]
由题设可知,积分与路径无关,于是有
\[
\frac{\partial Q\left(x,y\right)}{\partial x}=\frac{\partial P\left(x,y\right)}{\partial y}\Longrightarrow\left(x^2-y^2\right)f'\left(x^2-y^2\right)+f\left(x^2-y^2\right)=1
\]
\hfill\dotfill 5分

记$t=x^2-y^2$,即微分方程
\[
tf'\left(t\right)+f\left(t\right)=1\Leftrightarrow\left(tf\left(y\right)\right)'=1\Rightarrow tf\left(t\right)=y+C
\]
又$f(1)=0$,可得$C=-1\textbf{,}f(t)=1-\frac{1}{t}$,从而
\[
f\left(x^2-y^2\right)=1-\frac{1}{x^2-y^2}
\]
\hfill\dotfill 8分
	\end{solution}
	\end{answer}



\ws {\textbf{解答题}}{(\textbf{本题满分14分})\\}\\\\
设$f(x)$在区间$[0,1]$上连续,且$1\leq f\left(x\right)\leq3$.证明:
\[
0\leq\upint_0^1{f\left(x\right)\mathrm{d}x\upint_0^1{\frac{1}{f\left(x\right)}\mathrm{d}x\leq\frac{4}{3}}}
\]
	\begin{answer}
	\begin{proof}
	由 Cauchy-Schwarz 不等式:
\[
\upint_0^1{f\left(x\right)\mathrm{d}x\upint_0^1{\frac{1}{f\left(x\right)}\mathrm{d}x\geq\left(\upint_0^1{\sqrt{f\left(x\right)}\sqrt{\frac{1}{f\left(x\right)}}}\mathrm{d}x\right)}^2}=1
\]
\hfill\dotfill 4分

又由基本不等式得:
\[
\upint_0^1{f\left(x\right)\mathrm{d}x\upint_0^1{\frac{3}{f\left(x\right)}}\mathrm{d}x}\le\frac{1}{4}\left(\upint_0^1{f\left(x\right)\mathrm{d}x+\upint_0^1{\frac{3}{f\left(x\right)}\mathrm{d}x}}\right)^2
\]
再由条件$1\le f\left(x\right)\le 3$,有$\left(f\left(x\right)-1\right)\left(f\left(x\right)-3\right)\le 0$,则
\[
f\left(x\right)+\frac{3}{f\left(x\right)}\le 4\Rightarrow\upint_0^1{\left(f\left(x\right)+\frac{3}{f\left(x\right)}\right)\mathrm{d}x\le 4}
\]
\hfill\dotfill 10分

即可得
\[
1\le\upint_0^1{f\left(x\right)\mathrm{d}x\int_0^1{\frac{1}{f\left(x\right)}\mathrm{d}x\le\frac{4}{3}}}
\]
\hfill\dotfill 14分\\
	\end{proof}
	\end{answer}
\ws { \textbf{解答题}}{( \textbf{本题满分12分})\\}\\\\
计算三重积分$ \upiiint_{\left(V\right)}{\left(x^2+y^2\right)}\mathrm{d}V
$,其中$(V)$是由$x^2+y^2+\left(z-2\right)^2\geq 4$,$x^2+y^2+\left(z-1\right)^2\leq9$及$z\geq0$所围成的空间图形.

	\begin{answer}
	\begin{solution}
(1)计算打球$(V_1)$的积分,利用球坐标换元,令
\[
\left(V_1\right):\left\{\begin{array}{l}
x=r\sin\varphi\cos\theta ,y=r\sin\varphi\sin\theta ,z-1=r\cos\varphi\\
0\leq r\leq 3,0\leq\varphi\leq\pi ,0\leq\theta\leq 2\pi\\
\end{array}\right. 
\]
于是有
\[\upiiint _ { \left( V _ { 1 } \right) } \left( x ^ { 2 } + y ^ { 2 } \right) \mathrm { d } V = \upint _ { 0 } ^ { 2 \pi } \mathrm { d } \theta \upint _ { 0 } ^ { \pi }\mathrm { d } \varphi \upint _ { 0 } ^ { 3 } r ^ { 3 } \sin ^ { 2 } \varphi r ^ { 2 } \sin \varphi = \frac { 8 } { 15 } \cdot 3 ^ { 5 } \pi\]
\hfill\dotfill 4分

(2)计算小球$(V_2)$的积分,利用球坐标换元,令
\[
\left(V_2\right):\left\{\begin{array}{l}
x=r\sin\varphi\cos\theta ,y=r\sin\varphi\sin\theta ,z-2=r\cos\varphi\\
0\leq r\leq 2,0\leq\varphi\leq\pi ,0\leq\theta\leq 2\pi\\
\end{array}\right. 
\]
于是有
\[\upiiint _ { \left( V _ { 2} \right) } \left( x ^ { 2 } + y ^ { 2 } \right) \mathrm { d } V = \upint _ { 0 } ^ { 2 \pi } \mathrm { d } \theta \upint _ { 0 } ^ { \pi }\mathrm { d } \varphi \upint _ { 0 } ^ { 2 } r ^ { 2 } \sin ^ { 2 } \varphi r ^ { 2 } \sin \varphi = \frac { 8 } { 15 } \cdot 2^ { 5 } \pi\]
\hfill\dotfill 8分

(3)计算大球$z=0$下部分的积分$V_3$,利用球坐标换元,令
\[
\left(V_3\right):\left\{\begin{array}{l}
x=r\cos\theta ,y=r\sin\theta ,1-\sqrt{9-r^2}\leq z\leq 0\\
0\leq r\leq 2\sqrt{2},0\leq\theta\leq 2\pi\\
\end{array}\right. 
\]
于是有
\begin{align*}
\upiiint_{\left(V_3\right)}{\left(x^2+y^2\right)\mathrm{d}V}&=\upiint_{r\leq 2\sqrt{2}}{r\mathrm{d}r}\mathrm{d}\theta\upint_{1-\sqrt{9-r^2}}^0{r^2\mathrm{d}z}\\
&=\upint_0^{2\pi}{\mathrm{d}\theta}\upint_0^{2\sqrt{2}}{r^3\left(\sqrt{9-r^2}-1\right)}\\
&=\left(124-\frac{2}{5}\cdot 3^5+\frac{2}{5}\right)\pi 
\end{align*}
综上所述有
\begin{align*}
\upiiint _ { ( V ) } \left( x ^ { 2 } + y ^ { 2 } \right) \mathrm { d } V& = \upiiint _ { \left( V _ { 1 } \right) } \left( x ^ { 2 } + y ^ { 2 } \right) \mathrm { d } V - \upiiint _ { \left( V _ { 2 } \right) } \left( x ^ { 2 } + y ^ { 2 } \right) \mathrm { d } V - \upiiint _ { \left( V _ { 3 } \right) } \left( x ^ { 2 } + y ^ { 2 } \right) \mathrm { d } V\\
&= \frac { 8 } { 15 } \cdot 3 ^ { 5 } \pi+\frac { 8 } { 15 } \cdot 2^ { 5 } \pi+\left(124-\frac{2}{5}\cdot 3^5+\frac{2}{5}\right)\pi\\
&=\frac { 256 } { 3 } \pi
\end{align*}
\hfill\dotfill 12分
	\end{solution}
	\end{answer}

\ws {\textbf{解答题}}{(\textbf{ 本题满分14}分)\\}\\\\
设$f(x,y)$在区域D内可微,且$\sqrt{\left(\frac{\partial f}{\partial x}\right)^2+\left(\frac{\partial f}{\partial y}\right)^2}\leq M$,$A\left(x_1,y_1\right),B\left(x_2,y_2\right)$是D内两点,线段AB包含在D内,证明:
\[
|f\left(x_1,y_1\right)-f\left(x_2,y_2\right)|\leq M|AB|
\]
其中$|AB|$表示线段$AB$的长度.
	\begin{answer}
	\begin{proof}
作辅助函数
\[
\varphi\left(t\right)=f\left(x_1+t\left(x_2-x_1\right)\cdot y_1+t\left(y_2-y_1\right)\right)
\]
\hfill\dotfill 2分

显然$\varphi(t)$在$[0,1]$可导,根据Lagrange中值定理,存在$c\in(0,1)$,使得
\[
\varphi\left(1\right)-\varphi\left(0\right)=\varphi '\left(c\right)=\frac{\partial f\left(u,v\right)}{\partial u}\left(x_2-x_1\right)+\frac{\partial f\left(u,v\right)}{\partial v}\left(y_2-y_1\right)
\]
\hfill\dotfill 8分

即可得到
\begin{align*}
\left|\varphi\left(1\right)-\varphi\left(0\right)\right|&=\left| f\left(x_2,y_2\right)-f\left(x_1,y_1\right)\right|\\
&=\left|\frac{\partial f\left(u,v\right)}{\partial u}\left(x_2-x_1\right)+\frac{\partial f\left(u,v\right)}{\partial v}\left(y_2-y_1\right)\right|\\
&\leq\sqrt{\left(\frac{\partial f\left(u,v\right)}{\partial u}\right)^2+\left(\frac{\partial f\left(u,v\right)}{\partial v}\right)^2}\cdot\sqrt{\left(x_2-x_1\right)^2+\left(y_2-y_1\right)^2}\\
&\leq M\left| AB\right|
\end{align*}
\hfill\dotfill 14分
	\end{proof}
	\end{answer}




\ws { \textbf{解答题}}{( \textbf{本题满分14分})\\}\\\\
证明:对于连续函数$f(x)>0$,有
\[
\ln\upint_0^1{f\left(x\right)\mathrm{d}x\geq\upint_0^1{\ln f\left(x\right)\mathrm{d}x}}
\]

	\begin{answer}
	\begin{proof}
由定积分定义,将$[0,1]$分$n$等分,可取$\Delta x=\frac{1}{n}$,由“算术平均数$\geq$几何平均数”得:
\[
\frac{1}{n}\sum_{k=1}^n{f\left(\frac{k}{n}\right)\geq\sqrt[n]{f\left(\frac{1}{n}\right)\cdots f\left(\frac{n}{n}\right)}}=\mathrm{exp}{\frac{1}{n}\sum_{k=1}^n{\ln f\left(\frac{k}{n}\right)}}
\]
\hfill\dotfill 4分
\[
\Rightarrow\upint_0^1{f\left(x\right)}\mathrm{d}x\geq \mathrm{exp}{\lim_{n\rightarrow\infty}\frac{1}{n}\sum_{k=1}^n{\ln f\left(\frac{k}{n}\right)}}=\mathrm{exp}{\upint_0^1{\ln f\left(x\right)}}\mathrm{d}x
\]
\hfill\dotfill 10分

然后两边取对数即证
\[
\ln\upint_0^1{f\left(x\right)\mathrm{d}x\geq\upint_0^1{\ln f\left(x\right)\mathrm{d}x}}
\]
\hfill\dotfill 14分

或者考虑令$g(x)=\mathrm {ln}x$,则$g'\left(x\right)= \frac{1}{x}$,$g''\left(x\right)=-\frac{1}{x^2}<0$,所以$g(x)$为凹函数,可由琴声不等式定理即证.\\
	\end{proof}
	\end{answer}




\ws { \textbf{解答题}}{( \textbf{本题满分14分})\\}\\\\
已知${a_k}$,${b_k}$是正数数列,且$b_{k+1}-b_k\geq\delta >0,k=1,2,\cdots $,$\delta$为一切常数,证明:若级数$\sum_{k=1}^{+\infty}{a_k}$收敛,则级数$\sum_{k=1}^{+\infty}{\frac{k\sqrt[k]{\left(a_1a_2\cdots a_k\right)\left(b_1b_2\cdots b_k\right)}}{b_{k+1}b_k}}$收敛.
	\begin{answer}
	\begin{proof}
令$S_k=\sum_{i=1}^k{a_i}b_i,a_kb_k=S_k-S_{k-1},S_0=0,a_k=\frac{s_k-S_{k-1}}{b_k},k=1,2,\cdots $
\hfill\dotfill 4分
\begin{align*} 
\sum_{k=1}^N{a}_k&		=\sum_{k=1}^N{\frac{S_k-S_{k-1}}{b_k}}=\sum_{k=1}^{N-1}{\left(\frac{S_k}{b_k}-\frac{S_k}{b_{k+1}}\right)}+\frac{S_N}{b_N}\\
 &=\sum_{k=1}^{N-1}{\frac{b_{k+1}-b_k}{b_kb_{k+1}}}S_k+\frac{S_N}{b_N}\geq\sum_{k=1}^{N-1}{\frac{\delta}{b_kb_{k+1}}}S_k
\end{align*}
所以$\sum_{k=1}^{\infty}{\frac{S_k}{b_kb_{k+1}}}$收敛.
\hfill\dotfill 10分

由算术-几何平均不等式得
\[
\sqrt[k]{\left(a_1a_2\cdots a_k\right)\left(b_1b_2\cdots b_k\right)}\leq\frac{a_1b_1+\cdots +a_kb_k}{k}=\frac{S_k}{k}
\]
\[
\sum_{k=1}^{\infty}{\frac{k\sqrt[k]{\left(a_1a_2\cdots a_k\right)\left(b_1b_2\cdots b_k\right)}}{b_{k+1}b_k}}\leq\sum_{k=1}^{\infty}{\frac{S_k}{b_kb_{k+1}}}
\]
故结论成立.

\hfill\dotfill 14分
	\end{proof}
	\end{answer}

\mbox{}


%试卷正文结束

\end{document}