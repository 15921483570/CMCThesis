\let\leq\leqslant\let\geq\geqslant
\def\sgn{\mathop{\rm sgn}}
\everymath{\displaystyle}

\newcommand{\ee}{\mathrm e}
\newcommand{\dd}{\,\mathrm{d}}
\newcommand{\textop}[1]{\relax\ifmmode\mathop{\text{#1}}\else\text{#1}\fi}
%%%%%%定义两个列格式,数学与非数学模式
\newcolumntype{Y}{>{\centering\arraybackslash$}X<{$}}
\newcolumntype{Z}{>{\centering\arraybackslash}X}
\newcolumntype{L}{>{\raggedright\arraybackslash}X}
\newcolumntype{R}{>{\raggedleft\arraybackslash}X}
%定义绝对值
\newcommand\abs[1]{\left| #1 \right|}
\makeatletter
\newcommand{\rmnum}[1]{\romannumeral #1}
\newcommand{\Rmnum}[1]{\expandafter\@slowromancap\romannumeral #1@}
\makeatother

%%%%%%%%%%%%%%%%%%%%%%%%%%%%%%%%%%%%%%%%%%%%%%

\newcommand{\chaoda}{\fontsize{55pt}{\baselineskip}\selectfont}
\newcommand{\chuhao}{\fontsize{42pt}{\baselineskip}\selectfont}     % 字号设置
\newcommand{\xiaochuhao}{\fontsize{36pt}{\baselineskip}\selectfont} % 字号设置
\newcommand{\yihao}{\fontsize{28pt}{\baselineskip}\selectfont}      % 字号设置
\newcommand{\erhao}{\fontsize{21pt}{\baselineskip}\selectfont}      % 字号设置
\newcommand{\xiaoerhao}{\fontsize{18pt}{\baselineskip}\selectfont}  % 字号设置
\newcommand{\sanhao}{\fontsize{15.75pt}{\baselineskip}\selectfont}  % 字号设置
\newcommand{\xiaosanhao}{\fontsize{15pt}{\baselineskip}\selectfont} % 字号设置
\newcommand{\sihao}{\fontsize{14pt}{\baselineskip}\selectfont}      % 字号设置
\newcommand{\xiaosihao}{\fontsize{12pt}{14pt}\selectfont}           % 字号设置
\newcommand{\wuhao}{\fontsize{10.5pt}{12.6pt}\selectfont}           % 字号设置
\newcommand{\xiaowuhao}{\fontsize{9pt}{11pt}{\baselineskip}\selectfont}   % 字号设置
\newcommand{\liuhao}{\fontsize{7.875pt}{\baselineskip}\selectfont}  % 字号设置
\newcommand{\qihao}{\fontsize{5.25pt}{\baselineskip}\selectfont}    % 字号设置

\everymath{\displaystyle}
\usepackage[thmmarks,amsmath]{ntheorem}
\theoremstyle{nonumberplain}
\theoremheaderfont{\bfseries}
\theorembodyfont{\normalfont}
{
	\theoremstyle{nonumberplain}%不带标号
	\theoremheaderfont{\bfseries}%证明题头加粗
	\theorembodyfont{\normalfont}
	%	\theorembodyfont{\songti}%楷书字体
	\theoremsymbol{\mbox{$\Box$}}%结束以后自动画出一个小方块
	\newtheorem{solution}{解.}%名字叫做"solution",会在题头自动写上证明
}
{
	\theoremstyle{nonumberplain}%不带标号
	\theoremheaderfont{\bfseries}%证明题头加粗
	\theorembodyfont{\normalfont}
	%	\theorembodyfont{\songti}%楷书字体
	\theoremsymbol{\mbox{$\blacksquare$}}%结束以后自动画出一个小黑色方块
	%	\theoremsymbol{\mbox{$\Box$}}%结束以后自动画出一个小方块
	\newtheorem{proof}{证明.}%名字叫做"proof",会在题头自动写上证明
}

\usepackage{listings}
\usepackage{fontspec}
\setmonofont{Consolas}
\setmainfont{Consolas}


\setCJKmainfont[
Extension          = .otf      ,
Path               = fonts/    ,
UprightFont        = *-Regular ,
BoldFont           = *-Bold    ,
ItalicFont         = *-Regular ,
BoldItalicFont     = *-Bold    ,
ItalicFeatures     = FakeSlant ,
BoldItalicFeatures = FakeSlant] {NotoSerifCJKsc}
\setCJKsansfont[
Extension          = .otf      ,
Path               = fonts/    ,
UprightFont        = *-Regular ,
BoldFont           = *-Bold    ,
ItalicFont         = *-Regular ,
BoldItalicFont     = *-Bold    ,
ItalicFeatures     = FakeSlant ,
BoldItalicFeatures = FakeSlant] {NotoSansCJKsc}
\setCJKmonofont[
Extension          = .otf      ,
Path               = fonts/    ,
UprightFont        = *-Regular ,
BoldFont           = *-Bold    ,
ItalicFont         = *-Regular ,
BoldItalicFont     = *-Bold    ,
ItalicFeatures     = FakeSlant ,
BoldItalicFeatures = FakeSlant]{NotoSansMonoCJKsc}  

% origin
%\setCJKmainfont[Path=fonts/,]{HYShuSongErJ}%\rmfamily
\setCJKfamilyfont{zhsong}[Path=fonts/,]{HYShuSongErJ}

%\setCJKsansfont[Path=fonts/,]{HYZhongHeiS}
\setCJKfamilyfont{zhhei}[Path=fonts/,]{HYZhongHeiS}

%\setCJKmonofont[Path=fonts/,]{HYFangSongS}
\setCJKmonofont[Path=fonts/,]{HYShuSongErJ}		% 打字机字体, 华文宋体更好看
\setCJKfamilyfont{zhhei}[Path=fonts/,]{HYFangSongS}

\setCJKfamilyfont{zhkai}[Path=fonts/,]{HYKaiTiS}
%
\setCJKfamilyfont{bs}[Path=fonts/,]{HYDaSongJ}
%
\setCJKfamilyfont{kd}{华文行楷}
\def\heiti{\CJKfamily{zhhei}}
\def\songti{\CJKfamily{zhsong}}
\def\fangsong{\CJKfamily{zhfs}}
\def\kaishu{\CJKfamily{zhkai}}
\def\bs{\CJKfamily{bs}}


\usepackage{fbox}