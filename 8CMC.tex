\documentclass[UTF8]{ctexart}
\usepackage{mtpro2}
\usepackage{latexsym,bm,amsmath,amssymb}
\usepackage{amsfonts}
\usepackage{mathrsfs}
\everymath{\displaystyle}
\usepackage{enumerate,enumitem}
\usepackage{pifont}%各种符号需调用的宏包,比如笔符号
\usepackage{times}%time字体宏包
\setmainfont{Times New Roman}%新罗马字体

%=====中文字体设置=====%
%\usepackage{xeCJK}%用 XeLaTeX 时,ctexart 会调 xeCJK;用 (pdf)LaTeX 时,它会调 CJK。不需要手工再调用。
\newcommand{\zhongsong}{\CJKfontspec{方正中宋简体}}%方正中宋
\setCJKfamilyfont{huawenxingkai}{华文行楷} \newcommand*{\xingkai}{\CJKfamily{huawenxingkai}}%华文行楷
\usepackage{zref-user,zref-lastpage}%使用zref宏包,引用数字标签值和LastPage标签
\usepackage[paperwidth=21cm,paperheight=29.7cm,top=2.5cm,bottom=3cm,left=2cm,right=2cm]{geometry}%A4纸张

%=====给全文加个边框=====%
\usepackage{tikz}
\usepackage{fancybox}
\fancyput(-1.00cm,-24.8cm){\tikz \draw[solid,line width=1pt](0,0) rectangle (1cm+\textwidth,\textheight+1.1cm);}
%-----页面风格------
\usepackage{fancyhdr}%页面风格宏包提供fancy页面风格
\pagestyle{fancy}%fancy页面风格%\fancyhf{}清除所有页眉页脚
\renewcommand{\headrulewidth}{0pt}%去掉页眉线
%\renewcommand{\footrulewidth}{1.05pt}%设置页脚线为1.05pt
\fancyfoot[CO,CE]{\vspace*{1mm}第\;\thepage\;页(共\;\,\zpageref{LastPage}\; 页)}

%试卷正文开始----------
\begin{document}\zihao{5}
	\begin{center}
		{\zihao{-2}\xingkai 第八届中国大学生数学竞赛决赛试题}\\[1.5mm]
		{\zihao{4} (数学类,2017年3月18日)}\\[2mm]
		 \zihao{4}(16数学$-$胡八一)\\[3mm]
		{\zihao{4}考试形式:\underline{\;闭卷\;}\quad\, 考试时间:\underline{\;\;\,180\;\;\,}分钟   \quad\, 满分:\underline{\;\;\,100\;\;\,}分}
	\end{center}
	\vspace*{-7mm}\rule{\textwidth}{0.1pt}%分隔线
	\newcommand{\blank}{\underline{\hspace{2cm}}}
	%==============================================================================%
	%--------------------------------------正文开始---------------------------------%
	%===============================================================================%
	\setlength{\topsep}{1ex}%列表到上下文的垂直距离
	\setlength{\itemsep}{-2mm}%条目间距
\begin{enumerate}[labelsep=-0.2em,leftmargin=2em,align=left]
\item[{\textbf{一、填空题}}] \hspace{2.5mm}{\textbf{(本题满分20分, 共4小题, 每小题5分)}} 
\begin{enumerate}[label={\arabic*.},labelsep=-1.7em,leftmargin=1.0em,align=left]
			\item 设 $x^4+3x^2+2x+1=0$ 的 $4$ 个根为 $\alpha_1,\alpha_2,\alpha_3,\alpha_4$. 则行列式 $\begin{vmatrix}\alpha_1&\alpha_2&\alpha_3&\alpha_4\\
			\alpha_2&\alpha_3&\alpha_4&\alpha_1\\
			\alpha_3&\alpha_4&\alpha_1&\alpha_2\\
			\alpha_4&\alpha_1&\alpha_2&\alpha_3
			\end{vmatrix}$\;=\;\underline{\hspace{2cm}}
			%\\[2mm]{\heiti 答案}:
			\item
			设 $a$ 为实数,关于 $x$ 的方程 $3x^4-8x^3-30x^2+72x+a=0$ 有虚根的充分必要条件是 $a$ \\
			满足\;\blank
			%\\[2mm]{\heiti 答案}:
			\item 计算曲面积分 $I=\iint_S\frac{ax\,\mathrm{d}y\,\mathrm{d}z+(x+a)^2\,\mathrm{d}x\,\mathrm{d}y}{\sqrt{x^2+y^2+z^2}}$ ($a>0$ 为常数),\\[2mm]其中 $S:z=-\sqrt{a^2-x^2-y^2}$, 取上侧. $I=\blank$
			%\\[2mm]{\heiti 答案}:
			\item 记两特征值为 $1$, $2$ 的 $2$ 阶实对称矩阵全体为 $\Gamma$. $\forall A\in\Gamma$ , $a_{21}$ 表示 $A$ 的 $(2,1)$ 位置元素. \\
			则集合 $\cup_{A\in\Gamma}\{a_{21}\}$ 的最小元 = \blank
			%\\[2mm]{\heiti 答案}:
		\end{enumerate}
\item[{ \textbf{二、}}] {\textbf{(本题 15 分)}} 在空间直角坐标系中设旋转抛物面 $\Gamma$ 的方程为 $z=\tfrac{1}{2}(x^2+y^2)$. 设 $P$ 为空间中的平面 , \\
		它交抛物面 $\Gamma$ 与曲线 $C$. 问: $C$ 是何种类型的曲线 ? 证明你的结论.
		
\item[{ \textbf{三、证明题}}] \hspace{2.5mm}{\textbf{(本题 15 分)}} 设 $n$ 阶方阵 $\text{\textbf{\textit{A}}}$, $\text{\textbf{\textit{B}}}$ 满足:  秩 $(\text{\textbf{\textit{ABA}}})=$ 秩 $(\text{\textbf{\textit{B}}})$. 证明:  $\text{\textbf{\textit{AB}}}$ 与 $\text{\textbf{\textit{BA}}}$ 相似.
\item[{ \textbf{四、}}] {\textbf{(本题 20 分)}} 对 $\mathbb{R}$ 上无穷次可微的 (复值) 函数 $\varphi(x)$, 称 $\varphi(x)\in\mathscr{S}$, 如果 $\forall m,k\geqslant0$ \\[1mm]
		成立 $\sup_{x\in\mathbb{R}}\left|x^m\varphi^{(k)}(x)\right|<+\infty$. 若 $f\in\mathscr{S}$, 可定义 $\hat{f}(x):=\int_{\mathbb{R}}\hat{f}(y)e^{-2\pi ixy}\,\mathrm{d}y,\quad (\forall x\in\mathbb{R}).$ \\[1mm]
		证明: $\hat{f}(x)\in\mathscr{S}$, 且 \[f(x)=\int_{\mathbb{R}}\hat{f}(y)e^{2\pi ixy}\,\mathrm{d}y,\quad \forall x\in\mathbb{R}\]
\item[{\textbf{五、}}] {\textbf{(本题 15 分)}} 设 $n>1$ 为正整数. 令 \[S_n=\left(\frac{1}{n}\right)^n+\left(\frac{2}{n}\right)^n+\cdots+\left(\frac{n-1}{n}\right)^n\]
		\begin{enumerate}[label={\arabic*.},labelsep=-1.7em,leftmargin=1.0em,align=left]
			\item 证明:数列 $S_n$ 单调增且有界, 从而极限 $\lim_{n\to\infty}S_n$ 存在.
			\item 求极限 $\lim_{n\to\infty}S_n$.
		\end{enumerate}
\item[{\textbf{ 六、}}] {\textbf{(本题 20 分)}} 求证: 常微分方程 $\frac{\mathrm{d}y}{\mathrm{d}x}=-y^3+\sin x,x\in[0,2\pi]$  有唯一的满足 $y(0)=y(2\pi)$ 的解.
	\end{enumerate}
	
	
%正文结束	
\end{document}
