% !TeX document-id = {a9da8c09-8ca2-42d8-97aa-816e53d9f143}
% !TeX TXS-program:compile = txs:///latexmk/{}[-xelatex -synctex=1 -interaction=nonstopmode -silent %.tex]

\documentclass[hideanswer=false,
enfont=newtxtext,
zhfont=empty,
mathfont=newtxmath,
]{cmcthesis}
% 是否隐藏答案, hideanswer = false,true
% 英文字体选择, enfont     = newtxtext,noto,empty
% 中文字体设置,  zhfont     = zhnoto,origin,empty
% 数学字体设置, mathfont   = newtxmath,unicode-math,mtpro2,empty
% 选择了 empty 方案的可以使用通用的方法自行设置
% 以上选项可以全部留空,调用默认的配置
% noto 无论英文还是中文字体比较全,如果启用它,记得要把字体放在 fonts/ 下面
% origin 是八一的配置方案,开启它就是之前的设置,字体可以安装使用,或者放在当前文件夹或者 fonts/ 下面使用
% newtxtext 和 newtxmath 配
% noto(enfont) 和 zhnoto(zhfont) 配
%\setmainfont{Times New Roman}
%\setmathfont[
%Extension    = .otf          ,
%Path         = fonts/        ,
%BoldFont     = XITSMath-Bold ,
%StylisticSet = 8]{XITSMath-Regular}% only when mathfont=unicode-math,具体字体可以自己选择

\cmcthesissetup{% key = value 设置处
	%
}
\let\leq\leqslant\let\geq\geqslant
\def\sgn{\mathop{\rm sgn}}
\everymath{\displaystyle}

\newcommand{\ee}{\mathrm e}
\newcommand{\dd}{\,\mathrm{d}}
\newcommand{\textop}[1]{\relax\ifmmode\mathop{\text{#1}}\else\text{#1}\fi}
%%%%%%定义两个列格式,数学与非数学模式
\newcolumntype{Y}{>{\centering\arraybackslash$}X<{$}}
\newcolumntype{Z}{>{\centering\arraybackslash}X}
\newcolumntype{L}{>{\raggedright\arraybackslash}X}
\newcolumntype{R}{>{\raggedleft\arraybackslash}X}
%定义绝对值
\newcommand\abs[1]{\left| #1 \right|}
\makeatletter
\newcommand{\rmnum}[1]{\romannumeral #1}
\newcommand{\Rmnum}[1]{\expandafter\@slowromancap\romannumeral #1@}
\makeatother

%%%%%%%%%%%%%%%%%%%%%%%%%%%%%%%%%%%%%%%%%%%%%%

\newcommand{\chaoda}{\fontsize{55pt}{\baselineskip}\selectfont}
\newcommand{\chuhao}{\fontsize{42pt}{\baselineskip}\selectfont}     % 字号设置
\newcommand{\xiaochuhao}{\fontsize{36pt}{\baselineskip}\selectfont} % 字号设置
\newcommand{\yihao}{\fontsize{28pt}{\baselineskip}\selectfont}      % 字号设置
\newcommand{\erhao}{\fontsize{21pt}{\baselineskip}\selectfont}      % 字号设置
\newcommand{\xiaoerhao}{\fontsize{18pt}{\baselineskip}\selectfont}  % 字号设置
\newcommand{\sanhao}{\fontsize{15.75pt}{\baselineskip}\selectfont}  % 字号设置
\newcommand{\xiaosanhao}{\fontsize{15pt}{\baselineskip}\selectfont} % 字号设置
\newcommand{\sihao}{\fontsize{14pt}{\baselineskip}\selectfont}      % 字号设置
\newcommand{\xiaosihao}{\fontsize{12pt}{14pt}\selectfont}           % 字号设置
\newcommand{\wuhao}{\fontsize{10.5pt}{12.6pt}\selectfont}           % 字号设置
\newcommand{\xiaowuhao}{\fontsize{9pt}{11pt}{\baselineskip}\selectfont}   % 字号设置
\newcommand{\liuhao}{\fontsize{7.875pt}{\baselineskip}\selectfont}  % 字号设置
\newcommand{\qihao}{\fontsize{5.25pt}{\baselineskip}\selectfont}    % 字号设置

\everymath{\displaystyle}
\usepackage[thmmarks,amsmath]{ntheorem}
\theoremstyle{nonumberplain}
\theoremheaderfont{\bfseries}
\theorembodyfont{\normalfont}
{
	\theoremstyle{nonumberplain}%不带标号
	\theoremheaderfont{\bfseries}%证明题头加粗
	\theorembodyfont{\normalfont}
	%	\theorembodyfont{\songti}%楷书字体
	\theoremsymbol{\mbox{$\Box$}}%结束以后自动画出一个小方块
	\newtheorem{solution}{解.}%名字叫做"solution",会在题头自动写上证明
}
{
	\theoremstyle{nonumberplain}%不带标号
	\theoremheaderfont{\bfseries}%证明题头加粗
	\theorembodyfont{\normalfont}
	%	\theorembodyfont{\songti}%楷书字体
	\theoremsymbol{\mbox{$\blacksquare$}}%结束以后自动画出一个小黑色方块
	%	\theoremsymbol{\mbox{$\Box$}}%结束以后自动画出一个小方块
	\newtheorem{proof}{证明.}%名字叫做"proof",会在题头自动写上证明
}


\begin{document}
	\cmcthesistitle{
		date= 2018年10月27号 \thinspace 9:00 - 11:30  ,
	}
	
\addvspace{1\bigskipamount}

为了方便统一 \LaTeX{} 格式,故推出此摸板。为了能够简单入门该文档,请阅读该说明文档。方便模板的更新与维护,个人的配置放到 \verb|settings.tex| 文件中,这样更新模板是只需替换旧的 \verb|cmcthesis.cls| 文件即可。自己的 \verb|settings.tex| 文件保留。

首先是设置文档的开头选项
\begin{lstlisting}[style=tex]
\documentclass[
	hideanswer=false,
	enfont=newtxtext,
	zhfont=empty,
	mathfont=newtxmath,
]{cmcthesis}
\end{lstlisting}
\verb|hideanswer=false,true| 两个选项可以选择一个。它决定了 \verb|answer| 环境里面的内容是否显示,以及开头的说明和页脚的内容。

例如对于以下内容:
\begin{lstlisting}[style=tex]
设 $\alpha\in\left(0,1\right)$,则$ \lim_{n\rightarrow +\infty}\left(\left(n+1\right)^{\alpha}-n^{\alpha}\right)=$\underline{\hspace{3em}}.
\begin{answer}
\begin{solution}
等价无穷小$\left(1+x\right)^{\alpha}-1\backsim\alpha x$,得
\[
\lim_{n\rightarrow\infty}\left(\left(n+1\right)^{\alpha}-n^{\alpha}\right)=\lim_{n\rightarrow\infty}n^{\alpha}\left(\left(1+1/n\right)^{\alpha}-1\right)=\lim_{n\rightarrow\infty}n^{\alpha}\times\frac{\alpha}{n}=0
\]
\end{solution}
\end{answer}
\end{lstlisting}
在 \verb|hideanswer=true| 的选择下,则在 \verb|answer| 环境里设置的答案是不会显示的,但是 \verb|hideanswer=false| 的选择下是会出现的。该选项旨在让试题和答案在同一文档中,而不需要切换文档。

对于 \verb|enfont| 选项在 \verb|newtxtext,noto,empty| 中选一个,决定使用的英文字体。如果不喜欢用 \verb|newtxtext| 和 \verb|noto| ,可以选择 \verb|empty| 使用原始的配置。当且仅当使用 \verb|empty| 选项时指定字体才不会造成奇怪的问题。使用之后可以用以下类似的语句在 \verb|settings.tex| 指定英文字体。
\begin{lstlisting}[style=tex]
\setmainfont{⟨font⟩}[⟨font features⟩]
\setsansfont{⟨font⟩}[⟨font features⟩]
\setmonofont{⟨font⟩}[⟨font features⟩]
\end{lstlisting}

中文字体的选择使用 \verb|zhfont| 来控制,可选方案有 \verb|zhnoto,origin,empty| 。\verb|zhnoto| 选项需要中文的 \verb|noto| 字体,可在群文件中下载,它与 \verb|enfont=noto| 配合使用。下载的字体应该放在 \verb|font/| 文件夹下以供使用。\verb|origin| 是八一的配置方案,开启它就是之前模板的设置,字体可以安装使用,或者放在当前文件夹或者 \verb|font/| 文件夹下面使用。如果启用的 \verb|empty| 选项,同理是用默认的设置,可以自己指定字体。用类似下面的方法指定:
\begin{lstlisting}[style=tex]
\setCJKmainfont{⟨font⟩}[⟨font features⟩]
\setCJKsansfont{⟨font⟩}[⟨font features⟩]
\setCJKmonofont{⟨font⟩}[⟨font features⟩]
\end{lstlisting}

数学字体的选项用 \verb|mathfont| 选项进行控制,可选择的有 \verb|newtxmath,unicode-math,| \verb|mtpro2,empty| 四个。\verb|newtxmath| 是与 \verb|enfont=newtxtext| 配合使用的。如果使用 \verb|mtpro2| 选项请自行安装该宏包。如果使用 \verb|unicode-math| 或者 \verb|empty| 选项可以自行指定字体方案。 \verb|empty| 的自由度更高。使用 \verb|unicode-math| 选项可以使用以下命令配置数学字体:
\begin{lstlisting}[style=tex]
\setmathfont{⟨font⟩}[⟨font features⟩]
\end{lstlisting}

如果想在试卷开头给出一些参考公式或者其它参考资料可以使用 \verb|material| 环境,使用示例如下:
\begin{lstlisting}[style=tex]
\begin{material}%[参考资料] %可选参数,可以改变默认的 “参考公式” 四个字
参考的内容啊
\end{material}
\end{lstlisting}
	
	{\bfseries 一、填空题:本大题共{\sffamily 14}小题,每小题{\sffamily 4}分,共计{\sffamily 16}分.请把答案填写在\CJKunderdot{答题卡相应的位置上}.}

\end{document}