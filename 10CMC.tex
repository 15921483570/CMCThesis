\documentclass[11pt,twoside]{ctexart}
\usepackage{mtpro2}
\usepackage{CJKnumb}
\usepackage{amsmath,amssymb}
\usepackage{calc}
\usepackage{intcalc}
\usepackage{ifthen}
\usepackage{zref-user}
\usepackage{zref-lastpage}
\usepackage{makecell}
\usepackage{interfaces-makecell}
\usepackage{dashrule}
\usepackage{parskip}
\usepackage[paperwidth=195mm,paperheight=270mm,left=30mm,right=25mm,top=20mm,bottom=20mm,includefoot]{geometry}
\usepackage{enumerate}
\usepackage{fancyhdr}
\usepackage{tcolorbox}
\usepackage{lastpage}
\pagestyle{fancy}

%用到的长度变量
\newlength{\wot}
\newlength{\wol}
\newlength{\gmw}
\newlength{\dl}

%长度变量的初始值
\settowidth{\wot}{复核人}
\setlength{\wol}{0.3pt}
\setlength{\gmw}{6em}
\setlength{\dl}{10em}

%页眉设置开始
\renewcommand{\headrulewidth}{0pt}

%装订线开始
\fancyheadoffset[OL,ER]{\gmw}
\fancyhead[OL]{
	\ifnum\intcalcMod{\value{page}}{4}=1
	\rotatebox{90}
	{\begin{minipage}{1.1\textheight}
			\begin{center}
				省市:\rule[-.2ex]{\dl}{\wol} 学校:\rule[-.2ex]{\dl}{\wol}  姓名:\rule[-.2ex]{\dl}{\wol} 准考证号:\rule[-.2ex]{\dl}{\wol}\\
				\tiny \hdashrule[-3ex]{\textheight}{\wol}{3pt}\\[\smallskipamount]
				\makebox[0.6\textheight][s]{装订线内不要答题}\\[-3\smallskipamount]
				\hdashrule[-3ex]{\textheight}{\wol}{3pt}
			\end{center}
	\end{minipage} }
	\fi
}
\fancyhead[ER]{
	\ifnum\intcalcMod{\value{page}}{4}=0
	\rotatebox{-90}
	{\begin{minipage}{1.1\textheight}
			\begin{center}
				\tiny \hdashrule[-3ex]{\textheight}{\wol}{3pt}\\[\smallskipamount]
				\makebox[0.6\textheight][s]{装订线内不要答题}\\[-3\smallskipamount]
				\hdashrule[-3ex]{\textheight}{\wol}{3pt}
			\end{center}
	\end{minipage} }
	\fi
}
%装订线结束
%页眉设置结束
%页脚设置开始
\renewcommand{\footrulewidth}{\wol}
\fancyfoot[C]{\heiti 微信公众号:八一考研数学竞赛\quad 第{\bf\thepage} 页 (共~{\bf\pageref{LastPage}}~页)}
%\fancyfoot[C]{\large{{ \textbf{微信公众号之八一考研数学竞赛}}\qquad 共\zpageref{LastPage}页\quad 第\thepage 页}}
\newcounter{ns}
\newcounter{ts}
\newcounter{nq}
\newcommand{\wns}{\stepcounter{ns}\CJKnumber{\thens}、}
\newcommand{\wq}{\stepcounter{nq}\thenq.}

%大题前计分表格
\newcommand{\tbs}{\begin{tabular}{|c|c|c|}\hline \makebox[\wot]{得分}&\makebox[\wot]{评卷人}&\makebox[\wot]{复核人}\\ \hline
		& &\\ \hline\end{tabular}}

%排版大题前计分表格,序号,题型,大题说明
\newcommand{\ws}[2]{\raisebox{-1ex}{\begin{minipage}[b]{4.6\wot}\tbs\end{minipage}}
	\begin{minipage}[t]{\textwidth-6\wot} {\heiti \wns #1 } #2 \end{minipage} }
\makeatletter
\zref@newprop{totalsections}[3]{\arabic{ns}}
\zref@addprop{LastPage}{totalsections}
\AtBeginDocument{
	\setcounter{ts}{\zref@extractdefault{LastPage}{totalsections}{3}} }
\makeatother
\linespread{1.618}
\newcommand{\D}{\,\mathrm{d}}
\newcommand{\E}{\mathrm{e}}
\newcommand{\dlim}{\displaystyle \lim }
\newcommand{\dint}{\displaystyle \int }
\newcommand{\sets}[1]{\{ #1 \}}

\setCJKfamilyfont{huawenxingkai}{华文行楷} \newcommand*{\xingkai}{\CJKfamily{huawenxingkai}}%华文行楷
\newcommand{\dis}{\displaystyle}
\newcommand{\Rank}{\mathrm{Rank}\mspace{1mu}}
\newcommand{\Sign}{\mathrm{Sign}\mspace{1mu}}
\newcommand{\rd}{\mspace{1mu}\mathrm{d}}
\newcommand{\diag}{\mathrm{diag}\mspace{1mu}}
\newcommand{\tr}{\mathrm{tr}\mspace{1mu}}
\newcommand{\var}{\mathrm{var}}
\renewcommand{\Re}{\operatorname{Re}}
\renewcommand{\Im}{\operatorname{Im}}

%直立积分号,需要mathabx宏包
\makeatletter
\def\upintkern@{\mkern-7mu\mathchoice{\mkern-2mu}{}{}{}}
\def\upintdots@{\mathchoice{\mkern-4mu\@cdots\mkern-4mu}%
	{{\cdotp}\mkern1.5mu{\cdotp}\mkern1.5mu{\cdotp}}%
	{{\cdotp}\mkern1mu{\cdotp}\mkern1mu{\cdotp}}%
	{{\cdotp}\mkern1mu{\cdotp}\mkern1mu{\cdotp}}}
\newcommand{\upiint}{\DOTSI\protect\UpMultiIntegral{2}}
\newcommand{\upiiint}{\DOTSI\protect\UpMultiIntegral{3}}
\newcommand{\upiiiint}{\DOTSI\protect\UpMultiIntegral{4}}
\newcommand{\upidotsint}{\DOTSI\protect\UpMultiIntegral{0}}
\newcommand{\UpMultiIntegral}[1]{%
	\edef\ints@c{\noexpand\upintop
		\ifnum#1=\z@\noexpand\upintdots@\else\noexpand\upintkern@\fi
		\ifnum#1>\tw@\noexpand\upintop\noexpand\upintkern@\fi
		\ifnum#1>\thr@@\noexpand\upintop\noexpand\upintkern@\fi
		\noexpand\upintop
		\noexpand\ilimits@
	}%
	\futurelet\@let@token\ints@a
}
\makeatother

\DeclareFontFamily{U}{mathx}{\hyphenchar\font45}
\DeclareFontShape{U}{mathx}{m}{n}{
	<->s * [0.8]
	mathx10
}{}
\DeclareSymbolFont{mathx}{U}{mathx}{m}{n}
\DeclareFontSubstitution{U}{mathx}{m}{n}
\DeclareMathSymbol{\upintop}{\mathop}{mathx}{'273}
%\DeclareMathSymbol{\upiint}{\mathop}{mathx}{'274}
%\DeclareMathSymbol{\upiiint}{\mathop}{mathx}{'275}
\DeclareMathSymbol{\upointop}{\mathop}{mathx}{'276}
\DeclareMathSymbol{\upoiint}{\mathop}{mathx}{'277}
\makeatletter
\newcommand{\upint}{\DOTSI\upintop\ilimits@}
\newcommand{\upoint}{\DOTSI\upointop\ilimits@}

%分数线延长mdugm
\newcommand{\zfrac}[2]{\dfrac{{\raisebox{-0.7mm}{$#1$}}}{\;{\raisebox{0.2mm}{$#2$}}\;}}
\newcommand{\bfrac}[2]{\dfrac{{\raisebox{-0.7mm}{$#1$}}}{{\raisebox{0.2mm}{$#2$}}}}

%======================
%试卷头开始
\begin{document}
	%试卷标题开始
	\begin{center}\vspace{3mm}
		{\xingkai \Large 第十届全国大学生数学竞赛预赛参考答案}\\[0.8mm]
		{ $\left(\text{非数学类, 2018年10月27日}\right)$}\\
	\end{center}
	
	%试卷标题结束
	
	%输出"绝密"字样
	{\vspace{-1.3mm}\heiti 绝密$\bigstar$启用前}\\[-4\bigskipamount]\\[-12mm]
	\begin{center}
		\vspace*{2mm}
		(16数学$-$胡八一)\\[3mm]
		{考试形式:\underline{~闭卷~}~\hspace{2mm}考试时间:\underline{~~150~~}分钟~\hspace{2mm}满分:~\underline{~~100~~}~分}\\
		
		%根据大题数目自动生成计分总表
		\newcounter{tc}
		\newcounter{tcsr}
		\setcounter{tc}{\value{ts}+3}
		\setcounter{tcsr}{\value{tc}-1}
		\arrayrulewidth=2\wol 
		
		\vspace*{3.5mm}
\begin{tabular}{|m{3em}<{\centering}|*{11}{m{3.5em}<{\centering}|}}\hline
         题~号 & 一 & 二 & 三  & 四 & 五 &六 &七  &总~~分 \\\hline
		 满~分 & 24 & 8 & 14  & 12 & 14  &14 &14 &100    \\\hline
	 	 得~分 &    &   &     &    &     &   &  &\rule{0pt}{8mm} \\\hline
	\end{tabular}
	\\\vspace*{-1.5mm}
	\begin{equation*}
	\begin{aligned}
	\mbox{注意:}
	&1.\,\mbox{所有答题都须写在试卷密封线右边,写在其他纸上一律无效}.\hspace{12.0cm}\\
	&2.\,\mbox{密封线左边请勿答题,密封线外不得有姓名及相关标记}.\\
	&3.\,\mbox{如答题空白不够,可写在当页背面,并标明题号}.\\[-2mm]
	\end{aligned}
	\end{equation*}	
\end{center}

%试卷头结束

\addvspace{1\bigskipamount}

\ws{\textbf{填空题}}{( \textbf{本题满分24分,每题6分})\\}\\\\
\wq 设 $\alpha\in\left(0,1\right),\textrm{则}\displaystyle \lim_{n\rightarrow +\infty}\left(\left(n+1\right)^{\alpha}-n^{\alpha}\right)=$\underline{\hspace{3em}}.\\
【答案】0.

【解析】等价无穷小$\left(1+x\right)^{\alpha}-1\backsim\alpha x$,得
\[\lim _ { n \rightarrow \infty } \left( ( n + 1 ) ^ { \alpha } - n ^ { \alpha } \right) = \lim _ { n \rightarrow \infty } n ^ { \alpha } \left( ( 1 + 1 / n ) ^ { \alpha } - 1 \right)=\lim_{ n \rightarrow \infty }n^{\alpha}\times \frac{\alpha}{n}=0 \]

\wq $\textrm{若曲线}y=f\left(x\right)\textrm{是由}\left\{\begin{array}{l}
x=t+\cos t\\
e^y+ty+\sin t=1\\
\end{array}\right.\textrm{确定,则此曲线在}t=0$ 对应点处的\\
切线方程为\underline{\hspace{3em}}.\\
【答案】$x+y-1=0$.

【解析】易知$t=0$处上的曲线为点$(1,0)$,即方程组对$t$求导得
\[
\frac{\mathrm{d}x}{\mathrm{d}t}=1-\sin t\textbf{,}\frac{\mathrm{d}y}{\mathrm{d}t}=-\frac{y+\cos t}{e^y+t}
\]
\[
\Rightarrow\frac{\mathrm{d}y}{\mathrm{d}t}=\frac{\mathrm{d}y}{\mathrm{d}t}/\frac{\mathrm{d}x}{\mathrm{d}t}=-\frac{y+\cos t}{\left(e^y+1\right)\left(1-\sin t\right)}\Rightarrow\frac{\mathrm{d}y}{\mathrm{d}x}|_{t=0}=-1
\]
故曲线在$t=0$对应点处的切线方程为$x+y-1=0$.


\wq $\displaystyle \upint{\displaystyle\zfrac{\ln\left(1+\sqrt{1+x^2}\right)}{\left(1+x^2\right)^{\frac{3}{2}}}}\mathrm{d}x=$\underline{\hspace{3em}}.\\
【答案】$\displaystyle \zfrac{x}{\sqrt{1+x^2}}\ln\left(x+\sqrt{1+x^2}\right)-\displaystyle \zfrac{1}{2}\ln\left(1+x^2\right)+C
$.

【解析】简单的凑微分,如下
\begin{align*}
\upint{\zfrac{\ln\left(x+\sqrt{1+x^2}\right)}{\left(1+x^2\right)^{\frac{3}{2}}}}\mathrm{d}x&=\upint{\ln\left(x+\sqrt{1+x^2}\right)}\mathrm{d}\left(\zfrac{x}{\sqrt{1+x^2}}\right)\\
&=\zfrac{x}{\sqrt{1+x^2}}\ln\left(x+\sqrt{1+x^2}\right)-\upint{\frac{x}{1+x^2}\mathrm{d}x}\\
&=\zfrac{x}{\sqrt{1+x^2}}\ln\left(x+\sqrt{1+x^2}\right)-\zfrac{1}{2}\ln\left(1+x^2\right)+C
\end{align*}

\wq $\displaystyle\lim_{x\rightarrow 0}\displaystyle\frac{1-\cos x\sqrt{\cos 2x}\sqrt[3]{\cos 3x}}{x^2}=$\underline{\hspace{3em}}.\\
【答案】3.

【解析】这题方法很多,简单的等价无穷小,或拆项、洛必达以及泰勒都可以.
\begin{align*}
\lim_{x\rightarrow 0}\frac{1-\cos x\sqrt{\cos 2x}\sqrt[3]{\cos 3x}}{x^2}&=\lim_{x\rightarrow 0}\left(\frac{1-\cos x}{x^2}+\frac{\cos x\left(1-\sqrt{\cos 2x}\sqrt[3]{\cos 3x}\right)}{x^2}\right)\\
&=\frac{1}{2}+\lim_{x\rightarrow 0}\frac{1-\sqrt{\cos 2x}\sqrt[3]{\cos 3x}}{\boldsymbol{x}^2}\\
&=\frac{1}{2}+\lim_{x\rightarrow 0}\left(\frac{1-\cos\sqrt{2\boldsymbol{x}}}{\boldsymbol{x}^2}+\frac{\cos\sqrt{2\boldsymbol{x}}\left(1-\sqrt[3]{\cos 3\boldsymbol{x}}\right)}{\boldsymbol{x}^2}\right)\\
&=\frac{1}{2}+\lim_{x\rightarrow 0}\left(\frac{1-\sqrt{1+\left(\cos 2\boldsymbol{x}-1\right)}}{\boldsymbol{x}^2}+\frac{1-\sqrt[3]{1+\left(\cos 3\boldsymbol{x}-1\right)}}{\boldsymbol{x}^2}\right)\\
&=\frac{1}{2}+\lim_{x\rightarrow 0}\frac{1-\cos 2\boldsymbol{x}}{2\boldsymbol{x}^2}+\lim_{x\rightarrow 0}\frac{1-\cos 3\boldsymbol{x}}{3\boldsymbol{x}^2}\\
&=3
\end{align*}



\newpage
\ws { \textbf{解答题}}{(\textbf{ 本题满分8分})\\}\\\\
设函数$f(t)$在$t\ne 0$时一阶连续可导,且$f(1)=0$,求函数$f(x^2-y^2)$,使得曲线积分$\displaystyle \upint_L{y\left(2-f\left(x^2-y^2\right)\right)}\mathrm {d}x+xf\left(x^2-y^2\right)\mathrm{d}y
$与路径无关,其中$L$为任一不与直线$y=\pm x$相交的分段光滑闭曲线.

【解析】记$\left\{ \begin{array} { l } { P ( x , y ) = y \left( 2 - f \left( x ^ { 2 } - y ^ { 2 } \right) \right) } \\
 {Q ( x , y ) = x + x f \left( x ^ { 2 } - y ^ { 2 } \right) } \end{array} \right.$,于是
\[\left\{ \begin{array} { l } { \zfrac { \partial P ( x , y ) } { \partial y } = 2 - f \left( x ^ { 2 } - y ^ { 2 } \right) + 2 y ^ { 2 } f ^ { \prime } \left( x ^ { 2 } - y ^ { 2 } \right) } \\ 
{ \zfrac { \partial Q ( x , y ) } { \partial x } = f \left( x ^ { 2 } + y ^ { 2 } \right) + 2 x ^ { 2 } f ^ { \prime } \left( x ^ { 2 } - y ^ { 2 } \right) } \end{array} \right.\]
由题设可知,积分与路径无关,于是有
\[\zfrac { \partial Q ( x , y ) } { \partial x } = \zfrac { \partial P ( x , y ) } { \partial y } \Longrightarrow \left( x ^ { 2 } - y ^ { 2 } \right) f ^ { \prime } \left( x ^ { 2 } - y ^ { 2 } \right) + f \left( x ^ { 2 } - y ^ { 2 } \right) = 1\]
\hfill\dotfill 5分

记$t=x^2-y^2$,即微分方程
\[t f ^ { \prime } ( t ) + f ( t ) = 1 \Leftrightarrow ( t f ( y ) ) ^ { \prime } =1 \Rightarrow tf(t)=y+C\]
又$f(1)=0$,可得$C=-1\textbf{,}f(t)=1-\zfrac{1}{t}$,从而
\[
f\left(x^2-y^2\right)=1-\zfrac{1}{x^2-y^2}
\]
\hfill\dotfill 8分
\newpage




\ws {\textbf{解答题}}{(\textbf{本题满分14分})\\}\\\\
设$f(x)$在区间$[0,1]$上连续,且$1\leq f\left(x\right)\leq3$.证明:
\[
0\leq\upint_0^1{f\left(x\right)\mathrm{d}x\upint_0^1{\frac{1}{f\left(x\right)}\mathrm{d}x\leq\zfrac{4}{3}}}
\]
【证明】由 Cauchy-Schwarz 不等式:
\[
\upint_0^1{f\left(x\right)\mathrm{d}x\upint_0^1{\zfrac{1}{f\left(x\right)}\mathrm{d}x\geq\left(\upint_0^1{\sqrt{f\left(x\right)}\sqrt{\zfrac{1}{f\left(x\right)}}}\mathrm{d}x\right)}^2}=1
\]
\hfill\dotfill 4分

又由基本不等式得:
\[
\upint_0^1{f\left(x\right)\mathrm{d}x\upint_0^1{\zfrac{3}{f\left(x\right)}}\mathrm{d}x}\le\zfrac{1}{4}\left(\upint_0^1{f\left(x\right)\mathrm{d}x+\upint_0^1{\zfrac{3}{f\left(x\right)}\mathrm{d}x}}\right)^2
\]
再由条件$1\le f\left(x\right)\le 3$,有$\left(f\left(x\right)-1\right)\left(f\left(x\right)-3\right)\le 0$,则
\[
f\left(x\right)+\zfrac{3}{f\left(x\right)}\le 4\Rightarrow\upint_0^1{\left(f\left(x\right)+\zfrac{3}{f\left(x\right)}\right)\mathrm{d}x\le 4}
\]
\hfill\dotfill 10分

即可得
\[
1\le\upint_0^1{f\left(x\right)\mathrm{d}x\int_0^1{\zfrac{1}{f\left(x\right)}\mathrm{d}x\le\zfrac{4}{3}}}
\]
\hfill\dotfill 14分\\
 
\ws { \textbf{解答题}}{( \textbf{本题满分12分})\\}\\\\
计算三重积分$\displaystyle
\upiiint_{\left(V\right)}{\left(x^2+y^2\right)}\mathrm{d}V
$,其中$(V)$是由$x^2+y^2+\left(z-2\right)^2\geq 4$,$x^2+y^2+\left(z-1\right)^2\leq9$及$z\geq0$所围成的空间图形.

【解析】(1)计算打球$(V_1)$的积分,利用球坐标换元,令
\[\left( V _ { 1 } \right) : \left\{ \begin{array} { l } { x = r \sin \varphi \cos \theta , y = r \sin \varphi \sin \theta , z - 1 = r \cos \varphi } \\
 { 0 \leq r \leq 3,0 \leqslant \varphi \leq \pi , 0 \leqslant \theta \leq 2 \pi } \end{array} \right.\]
于是有
\[\upiiint _ { \left( V _ { 1 } \right) } \left( x ^ { 2 } + y ^ { 2 } \right) \mathrm { d } V = \upint _ { 0 } ^ { 2 \pi } \mathrm { d } \theta \upint _ { 0 } ^ { \pi }\mathrm { d } \varphi \upint _ { 0 } ^ { 3 } r ^ { 3 } \sin ^ { 2 } \varphi r ^ { 2 } \sin \varphi = \zfrac { 8 } { 15 } \cdot 3 ^ { 5 } \pi\]
\hfill\dotfill 4分

(2)计算小球$(V_2)$的积分,利用球坐标换元,令
\[\left( V _ { 2 } \right) : \left\{ \begin{array} { l } { x = r \sin \varphi \cos \theta , y = r \sin \varphi \sin \theta , z - 2 = r \cos \varphi } \\ 
{ 0 \leq r \leq 2,0 \leq \varphi \leq \pi , 0 \leq \theta \leq 2 \pi } \end{array} \right.\]
于是有
\[\upiiint _ { \left( V _ { 2} \right) } \left( x ^ { 2 } + y ^ { 2 } \right) \mathrm { d } V = \upint _ { 0 } ^ { 2 \pi } \mathrm { d } \theta \upint _ { 0 } ^ { \pi }\mathrm { d } \varphi \upint _ { 0 } ^ { 2 } r ^ { 2 } \sin ^ { 2 } \varphi r ^ { 2 } \sin \varphi = \zfrac { 8 } { 15 } \cdot 2^ { 5 } \pi\]
\hfill\dotfill 8分

(3)计算大球$z=0$下部分的积分$V_3$,利用球坐标换元,令
\[
\left(V_3\right):\left\{\begin{array}{l}
x=r\cos\theta ,y=r\sin\theta ,1-\sqrt{9-r^2}\leq z\leq 0\\
0\leq r\leq 2\sqrt{2},0\leq\theta\leq 2\pi\\
\end{array}\right. 
\]
于是有
\begin{align*}
\upiiint_{\left(V_3\right)}{\left(x^2+y^2\right)\mathrm{d}V}&=\upiint_{r\leq 2\sqrt{2}}{r\mathrm{d}r}\mathrm{d}\theta\upint_{1-\sqrt{9-r^2}}^0{r^2\mathrm{d}z}\\
&=\upint_0^{2\pi}{\mathrm{d}\theta}\upint_0^{2\sqrt{2}}{r^3\left(\sqrt{9-r^2}-1\right)}\\
&=\left(124-\frac{2}{5}\cdot 3^5+\frac{2}{5}\right)\pi 
\end{align*}
综上所述有
\begin{align*}
\upiiint _ { ( V ) } \left( x ^ { 2 } + y ^ { 2 } \right) \mathrm { d } V& = \upiiint _ { \left( V _ { 1 } \right) } \left( x ^ { 2 } + y ^ { 2 } \right) \mathrm { d } V - \upiiint _ { \left( V _ { 2 } \right) } \left( x ^ { 2 } + y ^ { 2 } \right) \mathrm { d } V - \upiiint _ { \left( V _ { 3 } \right) } \left( x ^ { 2 } + y ^ { 2 } \right) \mathrm { d } V\\
&= \zfrac { 8 } { 15 } \cdot 3 ^ { 5 } \pi+\zfrac { 8 } { 15 } \cdot 2^ { 5 } \pi+\left(124-\frac{2}{5}\cdot 3^5+\frac{2}{5}\right)\pi\\
&=\zfrac { 256 } { 3 } \pi
\end{align*}
\hfill\dotfill 12分
\newpage






\ws {\textbf{解答题}}{(\textbf{ 本题满分14}分)\\}\\\\
设$f(x,y)$在区域D内可微,且$\sqrt{\left(\displaystyle\frac{\partial f}{\partial x}\right)^2+\left(\displaystyle\zfrac{\partial f}{\partial y}\right)^2}\leq M$,$A\left(x_1,y_1\right),B\left(x_2,y_2\right)$是D内两点,线段AB包含在D内,证明:
\[
|f\left(x_1,y_1\right)-f\left(x_2,y_2\right)|\leq M|AB|
\]
其中$|AB|$表示线段$AB$的长度.

【证明】作辅助函数
\[\varphi ( t ) = f \left( x _ { 1 } + t \left( x _ { 2 } - x _ { 1 } \right) \cdot y _ { 1 } + t \left( y _ { 2 } - y _ { 1 } \right) \right)\]
\hfill\dotfill 2分

显然$\varphi(t)$在$[0,1]$可导,根据Lagrange中值定理,存在$c\in(0,1)$,使得
\[\varphi ( 1 ) - \varphi ( 0 ) = \varphi ^ { \prime } ( c ) = \frac { \partial f ( u , v ) } { \partial u } \left( x _ { 2 } - x _ { 1 } \right) + \frac { \partial f ( u , v ) } { \partial v } \left( y _ { 2 } - y _ { 1 } \right)\]
\hfill\dotfill 8分

即可得到
\begin{align*}
\left|\varphi\left(1\right)-\varphi\left(0\right)\right|&=\left| f\left(x_2,y_2\right)-f\left(x_1,y_1\right)\right|\\
&=\left|\zfrac{\partial f\left(u,v\right)}{\partial u}\left(x_2-x_1\right)+\zfrac{\partial f\left(u,v\right)}{\partial v}\left(y_2-y_1\right)\right|\\
&\leq\sqrt{\left(\zfrac{\partial f\left(u,v\right)}{\partial u}\right)^2+\left(\zfrac{\partial f\left(u,v\right)}{\partial v}\right)^2}\cdot\sqrt{\left(x_2-x_1\right)^2+\left(y_2-y_1\right)^2}\\
\leq M\left| AB\right|
\end{align*}
\hfill\dotfill 14分




\ws { \textbf{解答题}}{( \textbf{本题满分14分})\\}\\\\
证明:对于连续函数$f(x)>0$,有
\[
\ln\upint_0^1{f\left(x\right)\mathrm{d}x\geq\upint_0^1{\ln f\left(x\right)\mathrm{d}x}}
\]

【证明】由定积分定义,将$[0,1]$分$n$等分,可取$\Delta x=\displaystyle \zfrac{1}{n}$,由“算术平均数$\geq$几何平均数”得:
\[
\zfrac{1}{n}\sum_{k=1}^n{f\left(\zfrac{k}{n}\right)\geq\sqrt[n]{f\left(\frac{1}{n}\right)\cdots f\left(\zfrac{n}{n}\right)}}=\mathrm{exp}{\zfrac{1}{n}\sum_{k=1}^n{\ln f\left(\zfrac{k}{n}\right)}}
\]
\hfill\dotfill 4分
\[
\Rightarrow\upint_0^1{f\left(x\right)}\mathrm{d}x\geq \mathrm{exp}{\lim_{n\rightarrow\infty}\zfrac{1}{n}\sum_{k=1}^n{\ln f\left(\zfrac{k}{n}\right)}}=\mathrm{exp}{\upint_0^1{\ln f\left(x\right)}}\mathrm{d}x
\]
\hfill\dotfill 10分

然后两边取对数即证
\[
\ln\upint_0^1{f\left(x\right)\mathrm{d}x\geq\upint_0^1{\ln f\left(x\right)\mathrm{d}x}}
\]
\hfill\dotfill 14分

或者考虑令$g(x)=\mathrm {ln}x$,则$g'\left(x\right)=\displaystyle \zfrac{1}{x}$,$g''\left(x\right)=-\displaystyle\zfrac{1}{x^2}<0$,所以$g(x)$为凹函数,可由琴声不等式定理即证.\\




\ws { \textbf{解答题}}{( \textbf{本题满分14分})\\}\\\\
已知${a_k}$,${b_k}$是正数数列,且$b_{k+1}-b_k\geqslant\delta >0,k=1,2,\cdots $,$\delta$为一切常数,证明:若级数$\displaystyle\sum_{k=1}^{+\infty}{a_k}$收敛,则级数$\displaystyle\sum_{k=1}^{+\infty}{\displaystyle\frac{k\sqrt[k]{\left(a_1a_2\cdots a_k\right)\left(b_1b_2\cdots b_k\right)}}{b_{k+1}b_k}}$收敛.

【证明】令$S _ { k } = \displaystyle\sum _ { i = 1 } ^ { k } a _ { i } b _ { i } , a _ { k } b _ { k } = S _ { k } - S _ { k - 1 } , S _ { 0 } = 0 , a _ { k } = \displaystyle\zfrac { s _ { k } - S _ { k - 1 } } { b _ { k } } , k = 1,2 , \cdots$
\hfill\dotfill 4分
\begin{align*} 
	\sum _ { k = 1 } ^ { N } a _ { k } & = \sum _ { k = 1 } ^ { N } \zfrac { S _ { k } - S _ { k - 1 } } { b _ { k } } = \sum _ { k = 1 } ^ { N - 1 } \left( \zfrac { S _ { k } } { b _ { k } } - \zfrac { S _ { k } } { b _ { k + 1 } } \right) + \zfrac { S _ { N } } { b _ { N } } \\ 
	& = \sum _ { k = 1 } ^ { N - 1 } \zfrac { b _ { k + 1 } - b _ { k } } { b _ { k } b _ { k + 1 } } S _ { k } + \zfrac { S _ { N } } { b _ { N } } \geq \sum _ { k = 1 } ^ { N - 1 } \zfrac { \delta } { b _ { k } b _ { k + 1 } } S _ { k } 
\end{align*}
所以$\displaystyle\sum _ { k = 1 } ^ { \infty } \frac { S _ { k } } { b _ { k } b _ { k + 1 } }$收敛.
\hfill\dotfill 10分

由算术-几何平均不等式得
\[\sqrt [ k ] { \left( a _ { 1 } a _ { 2 } \cdots a _ { k } \right) \left( b _ { 1 } b _ { 2 } \cdots b _ { k } \right) } \leq \zfrac { a _ { 1 } b _ { 1 } + \cdots + a _ { k } b _ { k } } { k } = \zfrac { S _ { k } } { k }\]
\[\sum _ { k = 1 } ^ { \infty } \zfrac { k \sqrt [ k ] { \left( a _ { 1 } a _ { 2 } \cdots a _ { k } \right) \left( b _ { 1 } b _ { 2 } \cdots b _ { k } \right) } } { b _ { k + 1 } b _ { k } } \leqslant \sum _ { k = 1 } ^ { \infty } \zfrac { S _ { k } } { b _ { k } b _ { k + 1 } }\]
故结论成立.

\hfill\dotfill 14分
\mbox{}


%试卷正文结束
\end{document}
